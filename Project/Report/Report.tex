\documentclass[12pt]{report}
\usepackage{indentfirst}
\usepackage{graphicx}
\usepackage{float}
\usepackage[a4paper, total={7in, 10in}]{geometry}

\graphicspath{ {./Images/} }

\setlength{\parskip}{0.2em}
\renewcommand{\baselinestretch}{1.0}

\begin{document}

\title{\Huge{\textbf{FEUPBook}} \\ Relatório BDAD 2019/2020 \\ \Large{2MIEIC04}}
\author{Eduardo Correia \\ \texttt{up201806433} \and
	    Ricardo Fontão \\  \texttt{up201806317} \and
        João Diogo \\ \texttt{up201806779}}
\date{\today}

\begin{figure}[b] % Bottom of page
    \centering
    \includegraphics[width=0.8\textwidth]{feup-logo}
\end{figure}

\maketitle

\tableofcontents

\chapter{Introdução}

O nosso projeto consiste numa rede social denominada \textit{FEUPBook} (inspirado pela já existente rede social,  \textit{Facebook}). Nesta, utilizadores poderão criar uma conta pessoal para falar uns com os outros, através da troca de mensagens de texto e ver diversas publicações do seu interesse no seu \textit{feed} (a página inicial), bem como organizar \textbf{eventos}, juntarem-se em \textbf{grupos} ou edificar uma \textbf{página} dedicada a algum assunto em particular.

\chapter{Especificação} 

\section{Publicador}

Esta classe tem como função identificar quem são os elementos que podem realizar publicações, registar que publicações efetuaram e atribuir-lhes um \underline{nome}.

\subsection{Utilizador}

Esta classe representa um \textbf{membro} da rede social. Este possui um \underline{id} único para o identificar (correspondente ao último campo do \textit{url} do seu perfil. \par

Um utilizador possui ainda vários outros dados pessoais (que o utilizador pode optar por não preencher por motivos de privacidade): \underline{número de telemóvel}, \underline{data de nascimento}, \underline{género} (masculino ou feminino), morada (\underline{rua} e \underline{localização}). \par

Um \textbf{utilizador}, por norma, terá diversas \textbf{amizades} com outros utilizadores, para que possa aceder aos conteúdos do seu perfil, conversar mais facilmente com ele e ver as publicações no seu feed.\par

Para esse efeito, terá de enviar um pedido de amizade ao outro, o qual pode ser \textbf{aceite} ou não. Se for aceite, a ligação \textbf{pedido de amizade} deixa de existir e passa a ser uma ligação de \textbf{amizade}. Se for rejeitada, essa ligação deixa de existir sem ser substituída por nenhuma outra a um novo \textbf{pedido de amizade} surgir por parte de um dos dois utilizadores.

Tanto uma \textbf{amizade} como um \textbf{pedido de amizade} registam a \underline{data} em que foram efetuados.

\subsection{Página}

Esta classe representa uma \textbf{página} na nossa rede social. \par

Esta tem também associada a si várias publicações. \par

Os utilizadores relacionam-se com um página na medida em que um deles é o seu \underline{administrador} (normalmente correspondente ao criador da página, mas este cargo pode ser transferido) e possui vários \underline{seguidores}, utilizadores que colocaram gosto na página e vêm as publicações no feed.

Tem um \underline{utilizador} que é o seu administrador e pode ter muitos seguidores(utilizadores).

\section{Conversa}

Numa \textbf{rede social} é indispensável a capacidade de os utilizadores conversarem entre si, como tal estes podem agregar-se numa \underline{conversa} (que possui no mínimo 2 utilizadores). \par

Uma \textbf{conversa} é composta por diversas \underline{mensagens}. \par

Ainda neste contexto, cada utilizador pode possuir uma \underline{alcunha} própria.

\section{Multimédia}

A rede social do nosso projeto possui a capacidade de partilhar ficheiros \textbf{multimédia}, quer seja através de mensagens ou publicações e dividem-se essencialmente em três \textbf{categorias}, áudio, imagem e vídeo. Cada ficheiro destes possui um \underline{título}, correspondente ao seu nome em memória, bem como um \underline{url} que indica a localização do ficheiro para lhe aceder. \par

É possível, no entanto, enviar qualquer tipo de ficheiros, não só áudio, imagem ou vídeo, porém se a sua extensão não corresponder a nenhum deste tipo de ficheiros, será enviado como um ficheiro binário que o utilizador pode descarregar. \par

Existe ainda um limite máximo do tamanho ficheiro que o utilizador pode enviar e é de 25 MB. \par

Tanto o \textbf{áudio} como \textbf{vídeo} possuem um \underline{comprimento} da sua duração em segundos.

\section{Atividade}

Vários conteúdos na rede social terão \textbf{reações} por parte dos \textbf{utilizadores}, como tal, esta classe tem o objetivo de agregar esses mesmos conteúdos e manter o registo das reações que possuem. \par

Cada um desses conteúdos possui \underline{texto} a acompanhá-los e a \underline{data} em que foram efetuados.

\subsection{Publicação}

Não só com \textbf{mensagens} é possível comunicar na rede social, mas também com \textbf{publicações}. Possuem o benefício de não serem tão privadas (não precisam de ter um destinatário em particular, ficando publicadas no perfil do \textbf{publicador} em questão que seja o autor da \textbf{publicação}). \par

Esta pode ser composta por multimédia (áudio, foto ou vídeo) de tipos iguais ou diferentes. \par

Agrega ainda vários comentários feitos por utilizadores que pretendam demonstrar a sua opinião em relação à mesma. \par

\subsection{Mensagem}

Uma \textbf{mensagem} é o modo como os utilizadores comunicam numa \underline{conversa}. \par

Pode ser uma mensagem de texto, a partilha de um ficheiro multimédia ou ambos, ou seja, um utilizador pode enviar um ficheiro multimédia com descrição (texto), sem descrição, ou só uma mensagem de texto. \par

\subsection{Comentário}

Um \textbf{comentário} é feito por um \textbf{utilizador} a uma \textbf{publicação} como modo de iniciar uma discussão sobre a publicação ou simplesmente realizar algum tipo de observação/denotação. \par

\section{Reação}

Um utilizador nem sempre tem de se manifestar com mensagens ou comentários, podendo optar por simplesmente deixar uma \textbf{Reação}. Para tal, terá a opção de escolher uma das seguintes \textbf{reações}.

\begin{figure}[H]
    \centering
    \includegraphics[width=\textwidth]{reactions}
\end{figure}

\section{Evento}

Se um \textbf{utilizador} desejar, pode criar um \textbf{evento} para marcar um \textbf{acontecimento} relevante e convidar outros utilizadores a participarem no mesmo. \par

Um \textbf{evento} possui um \underline{nome} que o identifica, bem como uma breve \underline{descrição} do que se trata e, possivelmente, o local onde se realiza,r e a \underline{data} da sua realização.

Por definição, a reação padrão será o \textit{like}. \par

Um \textbf{utilizador} apenas pode reagir com uma das possíveis reações a um post ou comentários, mas estes podem ter reações de vários utilizadores, inclusive dos seus autores.

\subsection{Grupo}

Um grupo possui como função agregar diversos utilizadores com um interesse em comum. Um utilizador fazendo parte de um grupo, está habilitado a fazer uma publicação, estando esta visível para os restantes membros do grupo. Um grupo tem um utilizador como administrador.

\chapter{Modelo concetual}

\begin{figure}[h!]
    \centering
    \includegraphics[width=0.8\textwidth]{diagram}
\end{figure}

\chapter{Modelo relacional}

\section{Dependências funcionais}

\textbf{Publisher}(\underline{publisherID}, name)

publisherID $\rightarrow$ name

\vspace{2mm}

\textbf{User}(\underline{userID} $\rightarrow$ Publisher, phoneNumber, gender, birthDate, age, address)

userID $\rightarrow$ phoneNumber, gender, birthDate, age, address

birthDate $\rightarrow$ age

\vspace{2mm}

\textbf{Friendship}(\underline{senderID} $\rightarrow$ User, \underline{receiverID} $\rightarrow$ User, state, date)

senderID, receiverID $\rightarrow$ state, date

\vspace{2mm}

\textbf{Page}(\underline{pageID} $\rightarrow$ Publisher, website, adminID $\rightarrow$ User)

pageID $\rightarrow$ website, adminID

\vspace{2mm}

\textbf{PageFollower}(\underline{followerID} $\rightarrow$ User, pageID)

\vspace{2mm}

\textbf{Group}(\underline{groupID}, name, adminID $\rightarrow$ User)

groupID $\rightarrow$ name, adminID

\vspace{2mm}

\textbf{GroupMember}(\underline{memberID} $\rightarrow$ User, \underline{groupID} $\rightarrow$ Group)

\vspace{2mm}

\textbf{Chat}(\underline{chatID}, name)

chatID $\rightarrow$ name

\vspace{2mm}

\textbf{ChatParticipant}(\underline{participantID} $\rightarrow$ User, chatID $\rightarrow$ Chat, nickname)

participantID, chatID $\rightarrow$ nickname

\vspace{2mm}

\textbf{Multimedia}(\underline{multimediaID}, title, uri, size, format)

multimediaID $\rightarrow$ title, uri, size, format  

uri $\rightarrow$ title, size, format

\vspace{2mm}

\textbf{Audio}(\underline{audioID} $\rightarrow$ Multimedia, length)

audioID $\rightarrow$ length

\vspace{2mm}

\textbf{Image}(\underline{imageID} $\rightarrow$ Multimedia)

\vspace{2mm}

\textbf{Video}(\underline{videoID} $\rightarrow$ Multimedia, length)

videoID $\rightarrow$ length

\vspace{2mm}

\textbf{Activity}(\underline{activityID}, text, date)

activityID $\rightarrow$ text, date

\vspace{2mm}

\textbf{Message}(\underline{messageID} $\rightarrow$ Activity, dateSent, multimediaID $\rightarrow$ Multimedia, authorID $\rightarrow$ User, chatID $\rightarrow$ Chat)

messageID $\rightarrow$ dateSent, multimediaID, authorID, chatID

\vspace{2mm}

\textbf{Post}(\underline{postID} $\rightarrow$ Activity, publisherID $\rightarrow$ Publisher, multimediaID $\rightarrow$ Multimedia, pageID $\rightarrow$ Page, groupID $\rightarrow$ Group)

postID $\rightarrow$ publisherID, multimediaID, pageID, groupID

\vspace{2mm}

\textbf{Comment}(\underline{commentID} $\rightarrow$ Activity, authorID $\rightarrow$ User, postID $\rightarrow$ Post)

commentID $\rightarrow$ authorID, postID

\vspace{2mm}

\textbf{Reaction}(\underline{activityID} $\rightarrow$ Activity, \underline{userID} $\rightarrow$ User, type)

activityID, userID $\rightarrow$ type

\vspace{2mm}

\textbf{Event}(\underline{eventID}, name, description, occurenceDate, creatorID $\rightarrow$ User)

eventID $\rightarrow$ name, description, occurenceDate, creatorID

\vspace{2mm}

\textbf{EventParticipant}(\underline{participantID} $\rightarrow$ User, \underline{eventID} $\rightarrow$ Event)

\pagebreak

\section{Análise Forma Normal}

De acordo com o modelo relacional apresentado, todas as relações respeitam a forma normal de Boyce-Codd e,  consequentemente, a 3ª forma normal com a exceção da Relação User. \par
Em birthDate $\rightarrow$ age o elemento que se encontra do lado esquerdo da relação (birthDate) não é uma chave, constitundo deste modo uma violação à BCNF. No entanto, o lado direito da relação apenas é constituído por atributos primos fazendo com que respeite a 3ª forma normal, a qual exige que os atributos do lado esquerdo sejam chaves ou que o lado direito da relação seja apenas contituído por atributos primos. \par
Uma possível solução para este problema seria decompor User do seguinte modo:

\begin{itemize}
    \item User1(birthDate, age)
    birthDate $\rightarrow$ age
    \item User2(id, birthDate, phoneNumber, gender, address)
\end{itemize}

Quanto às restantes relações, uma vez que do lado esquerdo das dependências funcionais se encontram chaves primárias é possível concluir que respeitam a forma normal de Boyce-Codd, e como a 3ª forma normal é um 'super set' desta última também a irão respeitar.

\chapter{Restrições}

\section{Publisher}

\begin{itemize}
    \item \textit{publisherID} é a primary key (key restriction, PRIMARY KEY);
    \item \textit{name} é o nome do Publicador e não pode ser nulo (NOT NULL).
\end{itemize}

\section{User}

\begin{itemize}
    \item \textit{userID} é primary key (key restriction, PRIMARY KEY), é foreign key (referential integrity, FOREIGN KEY) e não pode ser igual a nenhum atributo \textit{pageID} de \textit{Page}.
    \item \textit{phoneNumber} é único a para cada utilizador (UNIQUE);
    \item \textit{gender} apenas pode ter os valores ‘M’ (masculino) e ‘F’(feminino) (CHECK);
    \item \textit{birthDate} não pode ter valores nulos (NOT NULL) e tem que originar uma idade superior a 13 anos (CHECK NOW() - birthDate / 365 $>$ 13)
\end{itemize}

\section{Page}

\begin{itemize}
    \item \textit{pageID} é a primary key (key restriction, PRIMARY KEY), é foreign key (referential integrity, FOREIGN KEY) e não pode ser igual a nenhum atríbuto idUtilizador de Utilizador.
\end{itemize}

\section{ChatParticipant}

\begin{itemize}
    \item (\textit{participantID}, \textit{chatID}) é a primary key (key restriction, PRIMARY KEY) and is a foreign key (referential integrity, FOREIGN KEY).
\end{itemize}

\section{Friendship}

\begin{itemize}
    \item (\textit{senderID}, \textit{receiverID}) é a primary key (key restriction, PRIMARY KEY), é foreign key (referential integrity, FOREIGN KEY) e têm que ser distintos;
    \item \textit{date} corresponde à data do envio do pedido de amizade CHECK(julianday(date) $\leq$ julianday('now')), tem de ser anterior ao momento atual e não pode ter valor nulo (NOT NULL);
    \item \textit{state} corresponde ao estado atual do pedido de amizade e apenas pode possuir os valores 1 (aceite), 2 (pendente) e 3 (rejeitada) (CHECK).
\end{itemize}

\section{EventParticipant}

\begin{itemize}
    \item (\textit{participantID}, \textit{eventID}) é a primary key (key restriction, PRIMARY KEY), é foreign key (referential integrity, FOREIGN KEY).
\end{itemize}

\section{PageFollower}

\begin{itemize}
    \item (\textit{followerID}, \textit{pageID}) é a primary key (key restriction, PRIMARY KEY), é foreign key (referential integrity, FOREIGN KEY).
\end{itemize}

\section{GroupMember}

\begin{itemize}
    \item (\textit{memberID}, \textit{groupID}) é a primary key (key restriction, PRIMARY KEY), é foreign key (referential integrity, FOREIGN KEY).
\end{itemize}

\section{Group}

\begin{itemize}
    \item \textit{groupID} é a primary key (key restriction, PRIMARY KEY);
    \item \textit{name} corresponde ao nome do grupo e não pode ter valor nulo (NOT NULL);
    \item \textit{adminID} é foreign key (referential integrity, FOREIGN KEY).
\end{itemize}

\section{Event}

\begin{itemize}
    \item \textit{eventID} é a primary key (key restriction, PRIMARY KEY);
    \item \textit{name}  corresponde ao \textit{name} do evento e não pode ter valor nulo (NOT NULL);
    \item descrição  corresponde à descrição do evento e não pode ter valor nulo (NOT NULL);
    \item \textit{occurenceDate} corresponde à data em que o evento se realiza e não pode ter valor nulo (NOT NULL);
    \item \textit{creatorID} é foreign key (referential integrity, FOREIGN KEY).
\end{itemize}

\section{Multimedia}

\begin{itemize}
    \item \textit{multimediaID} é primary key (key restriction, PRIMARY KEY);
    \item \textit{size} só pode tomar valores inteiros positivos (CHECK);
    \item \textit{type} pode tomar os seguintes valores: ".mp3", ".jpg", ".png", ".wav", ".mp4"... (CHECK).
\end{itemize}

\section{Audio}

\begin{itemize}
    \item \textit{audioID} é primary key (key restriction, PRIMARY KEY) e foreign key (referential integrity, FOREIGN KEY) e não pode ser igual a qualquer idImagem de Imagem e qualquer idVideo de Video;
    \item \textit{length} só pode tomar valores inteiros positivos (CHECK).
\end{itemize}

\section{Image}

\begin{itemize}
    \item \textit{imageID} é primary key (key restriction, PRIMARY KEY) e foreign key (referential integrity, FOREIGN KEY) e não pode ser igual a qualquer idAudio de Audio e qualquer idVideo de Video.
\end{itemize}

\section{Video}

\begin{itemize}
    \item \textit{videoID} é primary key (key restriction, PRIMARY KEY) e foreign key (referential integrity, FOREIGN KEY) e não pode ser igual a qualquer idImagem de Imagem e qualquer idAudio de Audio;
    \item \textit{length} só pode tomar valores inteiros positivos (CHECK).
\end{itemize}

\section{Reaction}

\begin{itemize}
    \item \textit{activityID} é primary key (key restriction, PRIMARY KEY) e foreign key (referential integrity, FOREIGN KEY);
    \item \textit{userID} é foreign key (referential integrity, FOREIGN KEY);
    \item \textit{type} representa as formas que a reação pode tomar e pode ter os seguinter valores: 1, 2, 3, 4, 5, 6 (CHECK).
\end{itemize}

\section{Comment}

\begin{itemize}
    \item \textit{commentID} é primary key (key restriction, PRIMARY KEY) e foreign key (referential integrity, FOREIGN KEY);
    \item \textit{authorID} é foreign key (referential integrity, FOREIGN KEY);
    \item \textit{postID} é foreign key (referential integrity, FOREIGN KEY).
\end{itemize}

\section{Post}

\begin{itemize}
    \item \textit{postID} é primary key (key restriction, PRIMARY KEY) e foreign key (referential integrity, FOREIGN KEY);
    \item \textit{publisherID}, \textit{multimediaID}, \textit{pageID} e \textit{groupID} são foreign keys (referential integrity, FOREIGN KEY).
\end{itemize}
    
\section{Message}

\begin{itemize}
    \item \textit{messageID} é primary key (key restriction, PRIMARY KEY) e foreign key (referential integrity, FOREIGN KEY);
    \item \textit{multimediaID}, \textit{authorID} e \textit{chatID} é foreign key (referential integrity, FOREIGN KEY);
    \item \textit{dateSent} corresponde à data de envio de uma mensagem e não pode ter valor nulo (NOT NULL).
\end{itemize}

\section{Activity}

\begin{itemize}
    \item \textit{activityID} é primary key (key restriction, PRIMARY KEY);
    \item \textit{date} correponde à data em que a atividade foi realizada. (tem de ser antes do momento presente)
\end{itemize}

\section{Chat}

\begin{itemize}
    \item \textit{chatID} é primary key (key restriction, PRIMARY KEY);
    \item \textit{name} correponde ao nome da conversa e não pode ser nulo (NOT NULL).
\end{itemize}

\chapter{Interrogações}

\section{Publicação com o maior número de likes  }

\section{Reação mais comum por utilizador}

\section{Duração média dos ficheiros de áudio/vídeo}

\section{Número de mensagens trocadas por cada utilizador em conversas com mais de dois pairticipantes}

\section{Recomendação de amigos}

\section{Utilizadores que são administradores de uma página e de um grupo com o mesmo nome dessa página}

\section{Número de comentários feitos por dia, no último mês antes de um evento, que se referem ao mesmo}

\section{Utilizador que deram o maior número de likes às publicações de um outro}

\section{Grau de separação entre cada par de utilizadores}

\section{Maior "influenciador" da rede social}

Esta interrogação permite obter uma estimativa do maior "influenciador" da nossa rede social. Para o cálculo desta estimativa realizou-se a média aritmética entre o número total de amigos do utilizador, a quantidade de eventos passados a que foi ou eventos futuros que planeia ir e o número de comentários de outros utilizadores feitos nas suas publicações.
Deste modo, é obtido uma pontuação para cada utilizador que classifica o seu grau de influência. Quem possuir o maior \textit{score} será considerado o maior influenciador do FEUPBook.

\begin{equation}
    score = \frac{N_{amigos} + N_{eventos} + N_{comentarios}}{3}
\end{equation}
      
\chapter{Gatilhos}

\end{document}